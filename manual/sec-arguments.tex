\section{Command Line Arguments}
\label{sec:arguments}

\begin{itemize}

\item  {\bf -a}

Estimate model-averaged phylogeny for each active criterion. See Section \ref{sec:consensus} for more details.

\item  {\bf -AIC}

Calculate the Akaike Information Criterion. See Section \ref{sec:aic}.

\item  {\bf -AICc}

Calculate the corrected Akaike Information Criterion. See Section \ref{sec:aic}.

\item  {\bf -BIC}

Calculate the Bayesian Information Criterion. See Section \ref{sec:bic}.

\item  {\bf -DT}

Calculate the decision theory criterion. See Section \ref{sec:dt}.

\item  {\bf -c} confidenceInterval

Sets the confidence interval for the model selection process (default is 100).

\item  {\bf -d} inputFile

Sets the input data file. jModelTest makes use of the ALTER library for converting several alignment formats to PHYLIP.

\item  {\bf -dLRT}

Perform dynamical likelihood ratio tests. See Section \ref{sec:dlrt} for more details.

\item  {\bf -f}

Include models with unequals base frecuencies.

\item  {\bf -g} numberOfRateCategories

Include models with rate variation among sites and sets the number of categories. Usually 4 categories are enough.

\item  {\bf -getPhylip}

Converts the input file into phylip format and exits. For example, the following command will generate a new PHYLIP file named ``input.nex.phy''.
\begin{lstlisting}
$java -jar jModelTest.jar -d input.nex -getPhylip
\end{lstlisting}

\item  {\bf -G} threshold

Heuristic search. Requires a threshold > 0 (e.g., -G 0.1)

\item  {\bf -h} confidenceInterval

Sets the confidence level for the hLRTs (default is 0.01)

\item  {\bf -help}

Displays a help message

\item  {\bf -hLRT}

Perform hierarchical likelihood ratio tests. See Section \ref{sec:hlrt} for more details.

\item  {\bf -H}

Information criterion for clustering search (AIC, AICc, BIC). (e.g., -H AIC) (default is BIC)

\item  {\bf -i}

Include models with a proportion invariable sites.

\item  {\bf -machinesfile} machinesFile

Gets the processors per host from a machines file (for MPI execution).

\item  {\bf -n} logSuffix

Execution name appended to the log filenames. By default, current time is used: yyyyMMddhhmmss.

\item  {\bf -o} outputFile

Redirects the output to a file.

\item  {\bf -O} {ftvwxgp}

Sets the hypothesis order for the hLRTs (e.g., -hLRT -O gpftv) (default is ftvwxgp)
\begin{itemize}
\item {\bf f} frequencies
\item {\bf t} transition/transversion ratio
\item {\bf v} 2ti4tv for subst=3 / 2ti for subst>3
\item {\bf w} 2tv
\item {\bf x} 4tv
\item {\bf g} gamma
\item {\bf p} proportion of invariable sites
\end{itemize}

See Section \ref{sec:hlrt} for more details.

\item  {\bf -p}

Calculate the parameter importances. See Section \ref{sec:param-importances}.

\item  {\bf -r}

Backward selection for the hLRT (default is forward).

\item  {\bf -s} 3|5|7|11|203

Sets the number of substitution schemes.
\begin{itemize}
     \item {\bf 3} JC/F81, K80/HKY, SYM/GTR (used by default).
     \item {\bf 5} JC/F81, K80/HKY, TrNef/TrN, TPM1/TPM1uf, SYM/GTR.
     \item {\bf 7} JC/F81, K80/HKY, TrNef/TrN, TPM1/TPM1uf, TIM1ef/TIM1, TVMef/TVM, SYM/GTR.
     \item {\bf 11} All models defined in Table~\ref{table-models}.
     \item {\bf 203} All possible GTR submatrices.
\end{itemize}

\item  {\bf -S} NNI|SPR|BEST

Defines the tree topology search operation option for Maximum-Likelihood search: 
\begin{itemize}
     \item {\bf NNI} Nearest Neighbour Interchange (fast).
     \item {\bf SPR} Subtree Pruning and Regrafting (slower).
     \item {\bf BEST} Best of NNI and SPR (slowest option) (used by default).
\end{itemize}

\item  {\bf --set-local-config} configFile
\label{pp:args-config}

Allows the user to set a local configuration file in replacement of conf/jmodeltest.conf.
See Section~\ref{sec:config} for more details.

\item  {\bf --set-property} propertyName=propertyValue

Allows the user to set a especific value for a property in replacement of the existing parameter in conf/jmodeltest.conf.
See Section~\ref{sec:config} for more details.

e.g., --set-property log-dir=myHome/myLogDirectory

\item  {\bf -t} fixed|BIONJ|ML

Base tree for likelihood calculations (e.g., -t BIONJ):
\begin{itemize}
     \item {\bf fixed}  Fixed BIONJ topology from JC model
     \item {\bf BIONJ}  Neighbor-Joining topology for each model
     \item {\bf ML}     Maximum Likelihood topology for each model (default)
\end{itemize}

\item  {\bf -tr} numberOfThreads

Number of threads to execute (default is the number of logical processors in the machine).

\item  {\bf -u} treeFile

Fixed tree for likelihood calculations defined by the user. If a user tree is defined with this command, -t argument is ignored.

\item  {\bf -uLnL}

Calculate delta AIC,AICc,BIC against unconstrained likelihood.

\item  {\bf -v}

Do model averaging and parameter importances. See Section \ref{sec:model-averaging}.

\item  {\bf -w}

Prints out the PAUP block.

\item  {\bf -z}

Strict consensus type for model-averaged phylogeny (default is majority rule). See Section \ref{sec:consensus}.

\end{itemize}
